%%%%%%%%%%%%%%%%%%%%%%%%%%%%%%%%%%%%%%%%%%%%%%%%%%%%%%%%%%%%%%%%%%%%%%%%%%%%%%%%
% Packages
%%%%%%%%%%%%%%%%%%%%%%%%%%%%%%%%%%%%%%%%%%%%%%%%%%%%%%%%%%%%%%%%%%%%%%%%%%%%%%%%
\usetikzlibrary{calc}

\newlength{\bleedZ}
\setlength{\bleedZ}{\bleed-1.5mm} % to avoid corner point (after cut)
%
% Generates the colored margin at the top of each page
\backgroundsetup{
scale=1,
angle=0,
color=black,
opacity=1,
contents={%
\begin{tikzpicture}[overlay, remember picture]
%
% Top color bar
\path[fill=\chaptercolor] (current page.north west) rectangle (\paperwidth,-7.8mm-\bleed); % bar heigth
%
% layout guides (works only with \bleed set to 3mm)
%
%    \begin{tikzpicture}[remember picture, overlay]
%         \foreach \x in {0, 8,65,131,136,202,210} {
%             \draw[help lines,gray!20] ([xshift=\x mm+\bleed] current page.south west) --([xshift=\x mm+\bleed] current page.north west);
%         }
%         \foreach \y in {0,4.2,87.5,92,95.8, 98,117,131,140,148} {
%             \draw[help lines,gray!20] ([yshift=\y mm+\bleed] current page.south west) --([yshift=\y mm+\bleed] current page.south east);
%         }
%    \end{tikzpicture}
%
% bleed marks
%
    \begin{tikzpicture}[remember picture, overlay]
         % vertical bleed ticks
         \draw[help lines] ([xshift= \bleed] current page.south west) --([xshift= \bleed,yshift= \bleedZ] current page.south west);
         \draw[help lines] ([xshift= \bleed] current page.north west) --([xshift= \bleed,yshift=-\bleedZ] current page.north west);
         \draw[help lines] ([xshift=-\bleed] current page.south east) --([xshift=-\bleed,yshift= \bleedZ] current page.south east);
         \draw[help lines] ([xshift=-\bleed] current page.north east) --([xshift=-\bleed,yshift=-\bleedZ] current page.north east);
         % horizontal bleed ticks
         \draw[help lines] ([yshift= \bleed] current page.south west) --([yshift= \bleed,xshift= \bleedZ] current page.south west);
         \draw[help lines] ([yshift= \bleed] current page.south east) --([yshift= \bleed,xshift=-\bleedZ] current page.south east);
         \draw[help lines] ([yshift=-\bleed] current page.north west) --([yshift=-\bleed,xshift= \bleedZ] current page.north west);
         \draw[help lines] ([yshift=-\bleed] current page.north east) --([yshift=-\bleed,xshift=-\bleedZ] current page.north east);
    \end{tikzpicture}
\end{tikzpicture}
}}
%
% Horizontal caption base lines (used in photopages)
\newcommand{\captionbaselines}%
{%
%\draw[help lines,gray!20] ([yshift=10.5 mm+\bleed] current page.south west) --([yshift=10.5 mm+\bleed] current page.south east);
%\draw[help lines,gray!20] ([yshift=135.5 mm+\bleed] current page.south west) --([yshift=135.5 mm+\bleed] current page.south east);
\begin{tikzpicture}[remember picture, overlay]
    % vertical bleed ticks
    \draw[help lines] ([xshift= \bleed] current page.south west) --([xshift= \bleed,yshift= \bleedZ] current page.south west);
    \draw[help lines] ([xshift= \bleed] current page.north west) --([xshift= \bleed,yshift=-\bleedZ] current page.north west);
    \draw[help lines] ([xshift=-\bleed] current page.south east) --([xshift=-\bleed,yshift= \bleedZ] current page.south east);
    \draw[help lines] ([xshift=-\bleed] current page.north east) --([xshift=-\bleed,yshift=-\bleedZ] current page.north east);
    % horizontal bleed ticks
    \draw[help lines] ([yshift= \bleed] current page.south west) --([yshift= \bleed,xshift= \bleedZ] current page.south west);
    \draw[help lines] ([yshift= \bleed] current page.south east) --([yshift= \bleed,xshift=-\bleedZ] current page.south east);
    \draw[help lines] ([yshift=-\bleed] current page.north west) --([yshift=-\bleed,xshift= \bleedZ] current page.north west);
    \draw[help lines] ([yshift=-\bleed] current page.north east) --([yshift=-\bleed,xshift=-\bleedZ] current page.north east);
\end{tikzpicture}
}
%
% Color definitions

%NOTE Colours seem to be defned elsewhere, unable to swap round Cayley and Wilson.....
\definecolor{colorpre}{RGB}{230, 230, 230} % schedule
\definecolor{colorBlyth}{RGB}{191,229,243}      % Blyth       : Auditorium B
\definecolor{colorMelville}{RGB}{242,219,233}   % Melville    : Poster area
\definecolor{colorWilson}{RGB}{253, 246, 166}     % Cayley      : Auditorium C
\definecolor{colorAuditorium}{RGB}{163,234,193} % Auditorium
\definecolor{colorCayley}{RGB}{212, 211, 243}     % Wilson      : Conference room 6&7

\usepackage{etoolbox}
\newcommand{\basecolor}{%
    \ifcase\arabic{chapter}\or colorpre\or colorpre\or colorAuditorium\or colorBlyth\or colorCayley\or colorWilson\or colorAuditorium\or colorMelville\or colorBlyth\or colorCayley\or colorWilson\or colorAuditorium\or colorMelville\or colorBlyth\or colorCayley\or colorWilson\or colorpre\fi% define colors per chapter
}

%%% FROM 2017 BOOK ===================================
%\definecolor{colorpre}{RGB}{230, 230, 230} % schedule
%\definecolor{colorIH}{RGB}{163, 234, 193} % Loyd
%\definecolor{colorICS}{RGB}{242, 219, 233} % Prometheus
%\definecolor{colorMSE}{RGB}{191, 229, 243} % Ockels
%\definecolor{colorCO}{RGB}{253, 246, 166} % Payne
%\definecolor{colorBD}{RGB}{212, 211, 243} % Hardham
%\definecolor{colorHPM}{RGB}{255, 212, 178} % Dattler
%\definecolor{colorSC}{RGB}{242, 219, 233} % Prometheus
%\definecolor{colorRCW}{RGB}{191, 229, 243} % Ockels
%\definecolor{colorAMS}{RGB}{253, 246, 166} % Payne
%\definecolor{colorFN}{RGB}{212, 211, 243} % Hardham
%\definecolor{colorCD}{RGB}{230, 230, 230} % author index
%\definecolor{colorPOST}{RGB}{212,211,243}% Posters
%\definecolor{colorthirdteen}{RGB}{208,245,229}%
%\definecolor{colorfourteen}{RGB}{217,243,253}%
%\definecolor{colorfifteen}{RGB}{217,243,253}   % <<< expand list here for more chapters
%                                               % !!! if not enough colors provided, package xcolor throws exception
%\usepackage{etoolbox}
%\newcommand{\basecolor}{%
%    \ifcase\arabic{chapter}\or colorpre\or colorpre\or colorIH\or colorICS\or colorMSE\or colorCO\or colorBD\or colorHPM\or colorSC\or colorRCW\or colorAMS\or colorFN\or colorCD\or colorPOST\or colorthirdteen\or colorfourteen\or colorfifteen\fi% define colors per chapter
%}
 %%%====================================================FROM 2017 BOOK


%\newcommand{\chaptercolor}{\basecolor!20!white}% makes it pastell
\newcommand{\chaptercolor}{\basecolor}

%
% Remove chapter heading
\usepackage[explicit]{titlesec}
\titlespacing*{\chapter}{0pt}{-45.7pt}{0pt} % non-zero value adjusts vertical spacing when new chapter starts
\titleformat{\chapter}[display]
  {}{}{0pt}{\Huge}
\titleformat{\part}
  {}{}{0pt}{}

%
% New chapter definition
\newcommand{\session}[2]{%
\chapter{#1\hspace{0.7mm}\crule[#2]{}}
%\vspace*{5pt} % required to counteract \chapter vspacing
}

%
% Table of content
\usepackage{tocloft}
\newcommand{\crule}[2][black]{\textcolor{#1}{\rule[-.2\baselineskip]{\baselineskip}{\baselineskip}}}
\renewcommand{\contentsname}{}
\setlength{\cftbeforetoctitleskip}{-3em}
\setcounter{tocdepth}{0}
% remove chapter numbers (source: http://tex.stackexchange.com/questions/71123/remove-section-number-toc-entries-with-tocloft)
\makeatletter
\renewcommand{\cftchappresnum}{\begin{lrbox}{\@tempboxa}}
\renewcommand{\cftchapaftersnum}{\end{lrbox}}
\makeatother
\setlength{\cftchapnumwidth}{0pt}
\renewcommand{\cftchapfont}{\hfill\bfseries}
\renewcommand{\cftchapleader}{}

%
% Document font
\usepackage[default]{sourcesanspro}
%\usepackage[default,tabular]{sourcesanspro}
\usepackage[T1]{fontenc}
\usepackage{eurosym}
\usepackage[utf8]{inputenc}
\usepackage{amsmath,amssymb}
\usepackage{sansmath}
\sansmath

\usepackage{multicol}
\setlength{\columnsep}{5mm}

\usepackage{graphicx}
\usepackage{transparent}
\usepackage{parskip}
\setlength{\parskip}{0pt} % 1ex plus 0.5ex minus 0.2ex}
\setlength{\parindent}{0pt}
\usepackage{import}
\usepackage{ifoddpage}
\newcommand{\abstractlinespacing}{\setlength{\parskip}{.15cm}}
\newcommand{\regularlinespacing}{\setlength{\parskip}{0cm}}

\usepackage{enumitem}
\setlist[itemize]{leftmargin=*} % left align list items
\setlist[enumerate]{leftmargin=*} % left align list items
\usepackage{tcolorbox}

\newcommand{\squishlist}{
   \begin{list}{$\bullet$}
    { \setlength{\itemsep}{0pt}      \setlength{\parsep}{0pt}
      \setlength{\topsep}{0pt}       \setlength{\partopsep}{0pt}
      \setlength{\leftmargin}{1.5em} \setlength{\labelwidth}{1em}
      \setlength{\labelsep}{0.5em} } }

\newcommand{\squishend}{
    \end{list}  }

\providecommand\phantomsection{}
\usepackage[colorlinks,
			urlcolor=blue,
			citecolor=black,
			linkcolor=blue,
			bookmarks=false,
	        hypertexnames=true, hidelinks=true]{hyperref} % for printing add: hidelinks=true
\hypersetup{pdfauthor={Roland Schmehl, Oliver Tulloch},
            pdfsubject={Airborne Wind Energy},
            pdftitle={Airborne Wind Energy Conference 2019 Book of Abstracts},
            pdfkeywords={Airborne Wind Energy} {Kite Power} {AWEC} {2019} {Airborne Wind Energy Conference} {Glasgow} {Strathclyde} {University} {Technology} {Germany} {AWESCO} {Makani} {Ampyx} {Enerkite} {TwingTec} {eKite} {enerate}}

\usepackage[a-1b]{pdfx} % for archival purposes. Will ignore hypersetup

%\usepackage{makeidx}
\usepackage{imakeidx}
\usepackage[columns=3,initsep=0pt,justific=RaggedRight]{idxlayout}
\renewcommand{\indexname}{Author index}
\makeatletter
\renewenvironment{theindex}
               {\section*{\indexname}%
                \@mkboth{\MakeUppercase\indexname}%
                        {\MakeUppercase\indexname}%
                \thispagestyle{plain}\parindent\z@
                \parskip\z@ \@plus .3\p@\relax
                \columnseprule \z@
                \columnsep 35\p@
                \let\item\@idxitem}
               {}
\makeatother
\makeindex
%\makeindex[options=-s ./index] % use with \usepackage{makeidx}
% or use  "makeindex -s index.ist book" to generate index file

%
% Figure captions
%\usepackage{caption}
%\captionsetup[figure]{labelformat=empty,font={small,it},justification=raggedright,skip=3pt}
\newcommand{\awecaption}[1]{{\small\itshape #1\par}}

%
% Marginnotes
\usepackage[fulladjust]{marginnote}
%\renewcommand{\marginnotevadjust}{-3mm} % Globally adjusts top margin in marginnote !!! PROBLEM !!! -> lifts up entire margin
\renewcommand{\marginnotevadjust}{-\baselineskip}
\newlength{\logoboxheight}

%
% Center environment without vertical space
\newenvironment{nscenter}
 {\parskip=0pt\par\nopagebreak\centering}
 {\par\noindent\ignorespacesafterend}

%
% Set page numer on right side of footer !!!! WRONG !!!!
\usepackage{fancyhdr}
\fancypagestyle{plain}{%
    \renewcommand{\headrulewidth}{0pt}%
    \fancyhf{} %
    \fancyfoot[LE,RO]{\vspace{-5pt}\thepage}% the vspace lifts the page number back to the text area
}
\pagestyle{plain}

\definecolor{darkgray}{gray}{0.3}
% commands for comments or todo notices
\newcommand{\ptitle}[1]%
{% comment out following line to remove all comments in document
%\newline\textcolor{darkgray}{#1}
%\newline\itshape\textcolor{black}{#1}
}%
\newcommand{\ppage}[1]%
{% comment out following line to remove all comments in document
#1
}%

%%%%%%%%%%%%%%%%%%%%%%%%%%%%%%%%%%%%%%%%%%%%%%%%%%%%%%%%%%%%%%%%%%%%%%%%%%%%%%%%
% http://tex.stackexchange.com/questions/195832/replace-several-letters-in-math-font
%%%%%%%%%%%%%%%%%%%%%%%%%%%%%%%%%%%%%%%%%%%%%%%%%%%%%%%%%%%%%%%%%%%%%%%%%%%%%%%%
\DeclareUnicodeCharacter{20AC}{\euro}

\DeclareSymbolFont{Greekletters}{OT1}{iwona}{m}{n}
\DeclareSymbolFont{greekletters}{OML}{iwona}{m}{it}
\DeclareMathSymbol{\Delta}{\mathord}{Greekletters}{"01}
\DeclareMathSymbol{\Theta}{\mathord}{Greekletters}{"02}
\DeclareMathSymbol{\Lambda}{\mathord}{Greekletters}{"03}
\DeclareMathSymbol{\Xi}{\mathord}{Greekletters}{"04}
\DeclareMathSymbol{\Pi}{\mathord}{Greekletters}{"05}
\DeclareMathSymbol{\Sigma}{\mathord}{Greekletters}{"06}
\DeclareMathSymbol{\Upsilon}{\mathord}{Greekletters}{"07}
\DeclareMathSymbol{\Phi}{\mathord}{Greekletters}{"08}
\DeclareMathSymbol{\Psi}{\mathord}{Greekletters}{"09}
\DeclareMathSymbol{\Omega}{\mathord}{Greekletters}{"0A}

\DeclareMathSymbol{\alpha}{\mathord}{greekletters}{"0B}
\DeclareMathSymbol{\beta}{\mathord}{greekletters}{"0C}
\DeclareMathSymbol{\gamma}{\mathord}{greekletters}{"0D}
\DeclareMathSymbol{\delta}{\mathord}{greekletters}{"0E}
\DeclareMathSymbol{\epsilon}{\mathord}{greekletters}{"0F}
\DeclareMathSymbol{\zeta}{\mathord}{greekletters}{"10}
\DeclareMathSymbol{\eta}{\mathord}{greekletters}{"11}
\DeclareMathSymbol{\theta}{\mathord}{greekletters}{"12}
\DeclareMathSymbol{\iota}{\mathord}{greekletters}{"13}
\DeclareMathSymbol{\kappa}{\mathord}{greekletters}{"14}
\DeclareMathSymbol{\lambda}{\mathord}{greekletters}{"15}
\DeclareMathSymbol{\mu}{\mathord}{greekletters}{"16}
\DeclareMathSymbol{\nu}{\mathord}{greekletters}{"17}
\DeclareMathSymbol{\xi}{\mathord}{greekletters}{"18}
\DeclareMathSymbol{\pi}{\mathord}{greekletters}{"19}
\DeclareMathSymbol{\rho}{\mathord}{greekletters}{"1A}
\DeclareMathSymbol{\sigma}{\mathord}{greekletters}{"1B}
\DeclareMathSymbol{\tau}{\mathord}{greekletters}{"1C}
\DeclareMathSymbol{\upsilon}{\mathord}{greekletters}{"1D}
\DeclareMathSymbol{\phi}{\mathord}{greekletters}{"1E}
\DeclareMathSymbol{\chi}{\mathord}{greekletters}{"1F}
\DeclareMathSymbol{\psi}{\mathord}{greekletters}{"20}
\DeclareMathSymbol{\omega}{\mathord}{greekletters}{"21}
\DeclareMathSymbol{\varepsilon}{\mathord}{greekletters}{"22}
\DeclareMathSymbol{\vartheta}{\mathord}{greekletters}{"23}
\DeclareMathSymbol{\varpi}{\mathord}{greekletters}{"24}
\DeclareMathSymbol{\varrho}{\mathord}{greekletters}{"25}
\DeclareMathSymbol{\varsigma}{\mathord}{greekletters}{"26}
\DeclareMathSymbol{\varphi}{\mathord}{greekletters}{"27}

\DeclareSymbolFont{arrows}{OMS}{iwona}{m}{n}
\DeclareMathSymbol{\rightarrow}{\mathrel}{arrows}{"21} \let\to=\rightarrow

%%%%%%%%%%%%%%%%%%%%%%%%%%%%%%%%%%%%%%%%%%%%%%%%%%%%%%%%%%%%%%%%%%%%%%%%%%%%%%%%
% Commands
%%%%%%%%%%%%%%%%%%%%%%%%%%%%%%%%%%%%%%%%%%%%%%%%%%%%%%%%%%%%%%%%%%%%%%%%%%%%%%%%
%
% Presenter portrait in inner margin
\reversemarginpar
\newlength{\authorvsep}
\setlength{\authorvsep}{1.2mm} % length separating author name and address
\newlength{\addressvsep}
\setlength{\addressvsep}{3mm}  % length separating author affiliation and address
%
% Macro for typesetting the standard abstract
% argument #1: presenter profile photo top
% argument #2: presenter profile name + address middle
% argument #3: presenter profile logo bottom
% argument #4: abstract title
% argument #5: abstract text
\newcommand{\abstractpage}[5]
{
\marginnote{
\begin{nscenter}
#1
\end{nscenter}
}%
\marginnote{
\begin{nscenter}
#2
\end{nscenter}
%}[36.8mm]% vertical distance author name
}[35mm]% vertical distance author name
%
% Presenter logo cluster
\def\logobox{%
\vbox{%
\begin{nscenter}
#3
\end{nscenter}
}}
\settoheight{\logoboxheight}{\logobox}
\setlength{\logoboxheight}{120mm-\logoboxheight} % +.3\baselineskip (to correct for depth) +4mm (to lift logo)
%
\marginnote{
\begin{nscenter}
\logobox
\end{nscenter}
}[\logoboxheight]% vertical distance author name
% Abstract
\begin{nscenter}
#4
\end{nscenter}
%
%\vspace{\baselineskip}
%
\vspace{-.4cm}
\abstractlinespacing
#5
\regularlinespacing
%
\vfill
%
%\marginnote{\logobox}[-\logoboxheight]%
\newpage
}

% Bibliographic references
\newlength{\awereferencesskip}
\setlength{\awereferencesskip}{2mm}
%
% Macro for typesetting the list of references
% argument #1: reference list, individual references separated by empty line
\newcommand{\awereferences}[1]%
{\small\itshape
References:

#1
\vfill
}

%
% Macro for captions that position depending on odd and even page placement
%
% Example 1: positioning at bottom of page
%\photocaption{\fill}{\fill}{white}% oneside mode OR {twoside mode AND odd page}
%             {\fill}{0mm}{white}%   twoside mode AND even page
%             {Caption text.}
%
\newcommand{\photocaption}[7]%
{%
\checkoddpage\ifoddpageoroneside
% oneside mode OR {twoside mode AND odd page}
\vspace*{#1}
\parbox{\linewidth-#2}{\begin{flushright}\color{#3}\itshape #7\end{flushright}}\hspace*{#2} % regular case: caption flushed to right side
\vspace{5mm}
\else
% twoside mode AND even page
\vspace*{#4}
\hspace*{#5}\parbox{\linewidth-#5}{\begin{flushleft}\color{#6}\itshape #7\end{flushleft}}   % regular case: caption flushed to left side
\vspace{2.3mm}
\fi
}

\newcommand{\captionbox}[5]%
{%
	\transparent{#4}
	\begin{tcolorbox}[standard jigsaw,width=#1,arc=0mm,boxrule=0pt,left=1pt,top=1pt,right=1pt,bottom=1pt,colback=#2]
		\texttransparent{1.0}{\textcolor{#3}{#5}}
	\end{tcolorbox}
}
